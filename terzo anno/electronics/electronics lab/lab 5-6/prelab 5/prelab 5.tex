\documentclass{article}
\usepackage[utf8]{inputenc}
\usepackage{amsmath,amssymb}
\usepackage{graphicx}
\usepackage[a4paper,margin=0.9cm]{geometry}
\usepackage{multicol}
\usepackage{sectsty}

\graphicspath{ {.} }

\DeclareMathOperator{\ima}{Im}
\newcommand\tab[1][0,5cm]{\hspace*{#1}}

\begin{document}
	%\allsectionsfont{\small}
	%\scriptsize
	
	\mbox{}
	\vspace{10cm}
	\begin{center}
		\textbf{\Huge{Properties of the Differential Amplifier}}\\
		\bigskip
		\Large{Tommaso Bertelli}\\
		\bigskip
		\Large{CO-526-B - Electronics Lab}\\
		\bigskip
		\Large{Instructor Uwe Pagel}\\
		\bigskip
		\Large{1/12/2024}\\
	\end{center}
	\pagebreak
	
	\section{Introduction - Prelab}
		\subsection{Metal Oxide Semiconductor Field Effect Transistors (MOSFET)}
			\begin{enumerate}
				\item 
				Enhancement MOSFET:
				- The transistor is normally off when no gate voltage is applied.
				- A positive (for NMOS) or negative (for PMOS) gate voltage is required to induce a conductive channel and turn it on.
				- Commonly used in modern electronics due to its low power consumption in the off state.\\
				Depletion MOSFET:		
				- The transistor is normally on without any gate voltage applied.
				- A gate voltage opposite to the type of the MOSFET (negative for NMOS, positive for PMOS) is applied to turn it off.
				- Less common compared to enhancement-mode MOSFETs.
				\item 
				NMOS Transistor:
				
				- Built using n-type material as the channel.
				- Requires a positive voltage at the gate relative to the source to turn it on.
				- Typically faster and has better electron mobility than PMOS.
				- Used for high-speed and high-performance applications.
				PMOS Transistor:
				
				- Built using p-type material as the channel.
				- Requires a negative voltage at the gate relative to the source to turn it on.
				- Slower than NMOS due to lower hole mobility.
				- Often used for low-power applications.
			\end{enumerate}
		\subsection{MOSFET as Amplifier}
		\subsection{MOSFET as Switch}
		\begin{enumerate}
			\item When \(U_{in}\) is 0V the mosfet is off, no current flows so \(V_{RD}\) = 0V and \(V_{DS} = V_{DD}\) = 10V.
			\item  when \(U_{in}\) is 2.4V 
		\end{enumerate}
	
\end{document}
