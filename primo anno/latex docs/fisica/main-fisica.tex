\documentclass{article}
\usepackage[utf8]{inputenc}
\usepackage{amsmath,amssymb}
\usepackage{graphicx}
\usepackage[a4paper,margin=1.2cm]{geometry}
\usepackage{multicol}
\graphicspath{ {./img/} }

\DeclareMathOperator{\ima}{Im}


\begin{document}

\section{Calcolo vettoriale}

\[(\overline{a} \times \overline{b}) = \] 
\[
\begin{vmatrix}
	\hat{u}x & \hat{u}y & \hat{u}z \\
	ax       & ay       & az       \\
	bx       & by       & bz       \\ 
\end{vmatrix}
\]
\[
= \hat{u}x(aybz-byaz) -\hat{u}y(axbz-bxaz) +\hat{u}z(axby-bxay) 
\]\\
\[
\overline{a} \perp \overline{b} \iff \overline{a} \cdot \overline{b}
\]

\section{Cinematica del punto materiale}
\large{Moto rettilineo uniformemente accelerato (o uniforme)}
\[x=x_0+v_0(t-t_0)+\frac{1}{2}a(t-t_0)^2\]
\[v=v_0+a(t-t_0)\]
\[a=cost\]\\\\
\large{Moto circolare}
\[a_t=v'=s''\]
\[a_n=\frac{v^2}{\rho}\]

posizione: \[\overline{r}(t) = x(t)\hat{u}(x) + y(t)\hat{u}(t)\]
traiettoria: \[\overline{r} = r(t)\]   \[y = y(x)\]
legge oraria: \[s = s(t)\]
\\
\large{Velocità}
\[\overline{v}(t) = \frac{d}{dt}\overline{r}(t)\]
\[\overline{v}(t) = \frac{d}{dt}x(t) + \frac{d}{dt}y(t)\]
Componenti intrinseche: \[\overline{v}_s(t) = v_s(t)\hat{u}(t)\]
\[|\overline{v}_s(t)| = |v_s(t)|\]\\
\large{Accelerazione}
\[\overline{a}(t) = \frac{d}{dt}\overline{v}(t)\]
\[\overline{a}(t) = \frac{d}{dt}v_x(t)\hat{u}x + \frac{d}{dt}v_y(t)\hat{u}y\]
Componenti intrinseche: \[\overline{a}(t) = a(t)\hat{u}_t + a(t)\hat{u}_n\]
\[|\overline{a}_t| = \frac{d}{dt}v_s(t) \text{     } |\overline{a}_n| = \frac{v_s^2}{\rho}\]\\\\\\
\large{Metodo risolutivo generale per problemi di cinematica}
\begin{enumerate}
	\item Sistema di riferimento 
	\item Condizioni iniziali \(x_0, v_0, a_0, t_0 ...\)
	\item Equazione del moto (se nota)
	\item Risolvere il problema
\end{enumerate}
\large{Moto parabolico}


\section{Dinamica del punto materiale}
\[\overline{F} = m\overline{a}\]\\
\[
\begin{cases}
	F_{tot} x = ma_x \\
	F_{tot} y = ma_y   \\
	F_{tot} z = ma_z   \\
\end{cases}
\]\\\\
\large{Reazioni vincolari}
\[-\overline{F}p + \overline{N} = 0\]
\[\overline{F}p = m\overline{g}\]
\\\\
\large{Forza attrito}
\[|F_s| = |\mu_sN|\]
\[|F_d| = |\mu_dN|\]
\\\\
\large{Forza elastica}
\[\overline{F}_e = -K(x-x_0)\hat{u}x\]
\large{Legge oraria moto armonico}
\[x(t) = A cos(wt + \phi)\]
velocità
\[x(t) = -Aw sin(wt + \phi)\]
accellerazione
\[x(t) = -Aw^2 cos(wt + \phi)\]
equazione differenziale
\[\frac{d^2\theta}{dt}+\omega^2\theta=0\]

Con \(A, \phi\) condizioni iniziali e \(w^2 = \frac{K}{m}\) (in caso di molla)\\
\(w=\frac{2\pi}{T}\)\\
\(t=\frac{1}{f}\)\\
\section{lavoro e energia}
Lavoro
\[L=\int_{a}^{b}\overline{F}\cdot\overline{r}\]
Potenza istantanea \(P=\frac{dL}{dt}=\overline{F}\cdot\overline{v}\)\\
Potenza media \(P=\frac{L}{\delta t}\)\\\\
\subsection{Energia potenziale}
E gravitazionale \(E_p = mgh\)\\
E elastica \(\frac{1}{2}k\Delta x^2\)\\
\Large{Energia cinetica}\\
\(E_k= \frac{1}{2}mv^2\)
\section{Gravitazione}
\(F=G\frac{m_{2} m_{2}}{r^2}\)\\
\(E_p=-G\frac{m_{1}m_{2}}{r}\)
\section{sistemi non inerziali}
\[\overline{v}=\overline{v'}+\overline{v_t}+\overline{w}\times\overline{r'}\]
\[\overline{a}=\overline{a'}+\overline{a_t}+\overline{\alpha}\times\overline{r'}+\overline{\omega}\times\overline{\omega}\times\overline{r'}+2\overline{\omega}\times\overline{v'}\]
\(\overline{\alpha}=\frac{d\omega}{dt}\)

\section{statica dei fluidi}

\[p=\frac{dF}{dS}\] 
\[[p]=[pa]=\frac{[N]}{[m^2]}\]
\[\text{con z1 minore di z2}\]
\[p(z2) = p(z1) - pg(z2-z1)\]\\
\begin{center}
	\LARGE{Legge di stevino}
\end{center}
\[\text{con h che aumenta andando in profondità}\]
\[p(z) = p0 + mgh\]\\
\begin{center}
	\LARGE{Principio di Archimede}\\
	\Large{Un corpo immerso in un fluido riceve una spinta dal basso verso l'alto pari al \underline{peso} del volume del fluido spostato}
\end{center}











\end{document}