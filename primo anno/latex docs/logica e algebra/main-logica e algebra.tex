\documentclass{article}
\usepackage[utf8]{inputenc}
\usepackage{amsmath,amssymb}
\usepackage{graphicx}
\usepackage[a4paper,margin=1.2cm]{geometry}
\usepackage{multicol}
\graphicspath{ {./img/} }

\DeclareMathOperator{\ima}{Im}


\begin{document}
	
		
		\section{Proprietà relazioni}
		{\Large\[\text{Proprietà di una relazione binaria  } R\subseteq A \times A\]}
		\\\\
		{\large Seriale}\\\\
		\(\forall a \in A \  \exists b \in A (a,b) \in R\)\\\\
		Grafo: ogni vetice ha una freccia uscente\\
		Matrice: ogni riga ha almeno un "1"\\\\
		{\large Riflessiva}\\\\
		\(\forall a \in A \  (a,a) \in R\)\\\\
		Grafo: ogni vetice ha un cappio\\
		Matrice: sulla diagonale ho tutti "1"\\\\
		{\large Simmetrica}\\\\
		\(\forall a,b \in A \  (a,b) \in R \Rightarrow (b,a)\in R\)\\\\
		Grafo: Ogni freccia in una direzione ne ha una della direzione opposta\\
		Matrice: \(Mr=Mr^T\)\\\\
		{\large Antisimmetrica}\\\\
		\(\forall a,b \in A \  se\ (a,b) \in R \ e\  (b,a)\in R \Rightarrow a=b\)\\\\
		Grafo: Eccetto i cappi, per ogni freccia non ce ne deve essere una uguale ma nella direzione opposta\\
		Matrice: eccetto la diagonale, se in pos (i,j) c'è un 1, allora in posizione (j,i) ci deve essere 0\\\\
		{\large Transitiva}\\\\
		\(\forall a,b,c \in A \  (a,b) \in R e\ (b,c)\in R \Rightarrow (a,c)\in R\)\\\\
		Grafo: se a è collegato a b e b è collegato a c anche a deve essere collegato a c \\\\\\\\
		
		{\large Osservazioni}
		\begin{itemize}
			\item seriale \(\nRightarrow\) riflessiva
			\item antisimmetrica \(\nRightarrow\) non simmetrica
			\item transitiva + simmetrica \(\nRightarrow\) riflessiva
			\item riflessiva \(\Rightarrow\) seriale
			\item transitiva + simmetrica + seriale \(\Rightarrow\) riflessiva
		\end{itemize}
		\ \\\\
		
		{\Large Relazioni di equivalenza}\\\\
		{\large Una relazione si dice di equivalenza se è riflessiva, transitiva, simmetrica}\\\\\\
	
		{\Large Relazioni d'ordine}\\\\
		{\large Una relazione si dice d'ordine se è riflessiva, transitiva, antisimmetrica}\\\\\\\\
		
		{\Large Elementi estremali}\\\\
		\begin{itemize}
			\item Massimo: \(se \ \forall \ x \in A \ a\leq x\)
			\item Minimo: \(se \ \forall \ x \in A \ a\geq x\)
			\item Minimale: \(\forall \ x \in A \ se\ x \leq a \Rightarrow x=a\)
			\item Massimale: \(\forall \ x \in A \ se\ x \geq a \Rightarrow x=a\)
		\end{itemize}
	 	\ \\
	 	Oss: Un minimo è minimale, un massimo è massimale (minimali e massimali esistono in relazioni d'ordine)\\\\
	 	{\large Maggiorante/minorante, sup/inf}\\
	 	Un elemento m si dice
	 	\begin{itemize}
	 		\item Maggiorante di B se \(\forall b \in B \  b \leq m\)
	 		\item Minorante di B se \(\forall b \in B \  b \geq m\)
	 		\item Estremo sup di B se è il minimo dei maggioranti (se esiste)
	 		\item Estremo inf di B se è il massimo dei minoranti (se esiste)
 		\end{itemize}
 		
	 	\section{Logica proposizionale}
	 	
	 	{\large Sintassi}
	 	\begin{itemize}
	 		\item Lettere enunciative: \( A_1, A_2, ... , A_n\) 
	 		\item Connettivi: \(\neg, \land, \lor, \implies, \iff \)
	 		\item Simboli ausiliari: \( (\ )\ ; \)
	 	\end{itemize}
 		\
 		{\large Formula ben formata}
 		\begin{enumerate}
 			\item Ogni lettera enunciativa è una f.b.f.
 			\item Se A, B sono sono f.b.f. allora \( (A\implies B),(A \iff B) , (A \land B), (A\lor B), (\neg A) \) sono f.b.f.
 			\item Nient'altro è una f.b.f.
 		\end{enumerate}
 		
 		{\large Priorità connettivi: \(\neg, \land, \lor, \implies, \iff \)}\\\\
 		Significato connettivi
 		\begin{itemize}
 			\item \( (A\implies B) \) Sempre vero se A=0, Se A=1 vero solo se anche B=1
 			\item \( (A\iff B) \) Vero se A=B
 		\end{itemize}
 		\
 		
 		\begin{itemize}
 			\item Una f.b.f. A si dice soddisfacibile se esiste almeno una interpretazione che è modello di A
 			\item Una f.b.f. A per cui ogni interpretazione è un modello si dice tautologia
 			\item Una f.b.f. che non ammette modelli si dice insoddisfacibile 
 			\item Una f.b.f. B che ha gli stessi modelli di A si dice conseguenza semantica di A
 		\end{itemize}
 		
 		{\large Risoluzione logica proposizionale}
 		
 		\begin{itemize}
 			\item Letterali: Una lettera enunciativa \((A)\) o la sua negata \((\neg A)\)
 			\item Clausola: Insieme di letterali (disgiunzione di letterali) \( (\{\neg A,B,C\},\{B,C,D\}) \)
 		\end{itemize}
 		
 		\begin{enumerate}
 			\item Portare in forma normale congiuntiva es: \( (A\lor B \lor \neg C) \land (B \lor D \lor \neg A)\) (or tra lettere e and tra gruppi)
 			\item Convertire a letterali e clausole es: \( \{A,B,\neg C\},\{B,D,\neg A\}\) (ogni parentesi diventa una clausola con i propri letterali dentro)
 			\item L'obiettivo è raggiungere la clausola vuota, abbinado una clausola con un'altra ed eliminando IL letterale che in una è normale e nell'altra è negato
 		\end{enumerate}
 		
 		
 		procedere con logica del primo ordine (forse meglio guardare prima esercitazione)
	 	
	 	
		
		
		
		
		
		
\end{document}