\documentclass{article}
\usepackage[utf8]{inputenc}
\usepackage{amsmath,amssymb}
\usepackage{graphicx}
\usepackage[a4paper,margin=0.9cm]{geometry}
\usepackage{multicol}
\usepackage{sectsty}

\graphicspath{ {./img/} }

\DeclareMathOperator{\ima}{Im}
\newcommand\tab[1][0,5cm]{\hspace*{#1}}

\begin{document}
		\allsectionsfont{\small}
		\scriptsize
		
		\begin{multicols*}{3}
		\section{Proprietà relazioni}
		
		\subsection{seriale}
		\(\forall a \in A \  \exists b \in A (a,b) \in R\)\\
		Grafo: ogni vetice ha una freccia uscente\\
		Matrice: ogni riga ha almeno un "1"
		\subsection{riflessiva}
		\(\forall a \in A \  (a,a) \in R\)\\
		Grafo: ogni vetice ha un cappio\\
		Matrice: sulla diagonale ho tutti "1"
		\subsection{simmetrica}
		\(\forall a,b \in A \  (a,b) \in R \Rightarrow (b,a)\in R\)\\
		Grafo: Ogni freccia in una direzione ne ha una della direzione opposta\\
		Matrice: \(Mr=Mr^T\)
		\subsection{antisimmetrica}
		\(\forall a,b \in A \  se\ (a,b) \in R \ e\  (b,a)\in R \Rightarrow a=b\)\\
		Grafo: Non ci devono essere doppie freccie\\
		Matrice: eccetto la diagonale, se in pos (i,j) c'è un 1, allora in posizione (j,i) ci deve essere 0
		\subsection{transitiva}
		\(\forall a,b,c \in A \  (a,b) \in R e\ (b,c)\in R \Rightarrow (a,c)\in R\)\\
		Grafo: se a è collegato a b e b è collegato a c anche a deve essere collegato a c \\
		Matrice: \(Mr^2\subseteq Mr\)\\
		
		\textbf{Osservazioni}
		\begin{itemize}
			\setlength\itemsep{0.1mm}
			\item seriale \(\nRightarrow\) riflessiva
			\item antisimmetrica \(\nRightarrow\) non simmetrica
			\item transitiva + simmetrica \(\nRightarrow\) riflessiva
			\item riflessiva \(\Rightarrow\) seriale
			\item transitiva + simmetrica + seriale \(\Rightarrow\) riflessiva
		\end{itemize}
		
		
		\subsection{Relazioni di equivalenza}
		Una relazione si dice di equivalenza se è riflessiva, transitiva, simmetrica (tutti i possibili collegamenti in ogni componente connessa nel grafo)
	
		\subsection{Relazioni d'ordine}
		Una relazione si dice d'ordine se è riflessiva, transitiva, antisimmetrica (per esistere una ch d'ordine la relazione deve essere antisimmetrica, se facendo la chiusura riflessiva e transitiva rimane antisimmetrica ora è una ch d'ordine)
		
		\subsection{elementi estremali}
		\begin{itemize}
			\setlength\itemsep{0.1mm}
			\item Massimo: \(se \ \forall \ x \in A \ a\leq x\)
			\item Minimo: \(se \ \forall \ x \in A \ a\geq x\)
			\item Minimale: \(\forall \ x \in A \ se\ x \leq a \Rightarrow x=a\)
			\item Massimale: \(\forall \ x \in A \ se\ x \geq a \Rightarrow x=a\)
		\end{itemize}
	 
	 	Oss: Un minimo è minimale, un massimo è massimale (minimali e massimali esistono in relazioni d'ordine)
	 	\subsection{Maggiorante/minorante, sup/inf}
	 	Un elemento m si dice
	 	\begin{itemize}
	 		\setlength\itemsep{0.1mm}
	 		\item Maggiorante di B se \(\forall b \in B \  b \leq m\)
	 		\item Minorante di B se \(\forall b \in B \  b \geq m\)
	 		\item Estremo sup di B se è il minimo dei maggioranti (se esiste)
	 		\item Estremo inf di B se è il massimo dei minoranti (se esiste)
 		\end{itemize}
 		
 		\subsection{funzioni in relazioni}
 		\textbf{Proprietà della funzionalità :}\\
 		Grafo: un elemento punta solo ad un altro (possono esserci varie funzioni da una relazione, ma la relazione deve essere per forza seriale)
 		Matrice: per avere una funz devo avere un 1 per riga \\\\
 		\textbf{Funzione iniettiva (ha inversa destra):}\\
 		Matrice: in ogni colonna c'è al più un 1
 		\textbf{Funzione suriettiva (ha inversa sinistra):}\\
 		Matrice: in ogni colonna c'è almeno un 1\\\\
 		
 		
 		
 		
	 	\section{Logica proposizionale}
	 	
	 	\subsection{sintassi}
	 	\begin{itemize}
	 		\setlength\itemsep{0.1mm}
	 		\item Lettere enunciative: \( A_1, A_2, ... , A_n\) 
	 		\item Connettivi: \(\neg, \land, \lor, \implies, \iff \)
	 		\item Simboli ausiliari: \( (\ )\ ; \)
	 	\end{itemize}
 		
 		\subsection{formula ben formata}
 		\begin{enumerate}
 			\item Ogni lettera enunciativa è una f.b.f.
 			\item Se A, B sono sono f.b.f. allora \( (A\implies B),(A \iff B) , (A \land B), (A\lor B), (\neg A) \) sono f.b.f.
 			\item Nient'altro è una f.b.f.
 		\end{enumerate}
 		
 		\textbf{ Priorità connettivi: \(\neg, \land, \lor, \implies, \iff \)}\\
 		\textbf{Significato connettivi}
 		\begin{itemize}
 			\setlength\itemsep{0.1mm}
 			\item \( (A\implies B) \) Sempre vero se A=0, Se A=1 vero solo se anche B=1
 			\item \( (A\iff B) \) Vero se A=B
 			\item  \( (A\implies B) = \neg A \lor B\)
 			\item \( (A\iff B) = (A\implies B)\land(B\implies A) \) 
 		\end{itemize}
 		\subsection{equivalenze}
 		\begin{itemize}
 			\setlength\itemsep{0.1mm}
 			\item \(A\implies B = \neg B \implies \neg A\)
 			\item \((\neg A \land A)\lor B = B\)
 			\item \(A \land (A \lor B) = A\)
 			\item \(A \lor (A \land B) = A\)
 			\item \(A \land (B \lor C) = (A \land B) \lor (A \land C)\)
 			\item \(A \lor (B \land C) = (A \lor B) \land (A \lor C)\)
 		\end{itemize}
 		
 		\begin{itemize}
 			\setlength\itemsep{0.1mm}
 			\item Una f.b.f. A si dice soddisfacibile se esiste almeno una interpretazione che è modello di A
 			\item Una f.b.f. A per cui ogni interpretazione è un modello si dice tautologia
 			\item Una f.b.f. che non ammette modelli si dice insoddisfacibile 
 			\item Una f.b.f. B che ha gli stessi modelli di A si dice conseguenza semantica di A
 		\end{itemize}
 		
 		\subsection{risoluzione logica proposizionale}
 		
 		\begin{itemize}
 			\setlength\itemsep{0.1mm}
 			\item Letterali: Una lettera enunciativa \((A)\) o la sua negata \((\neg A)\)
 			\item Clausola: Insieme di letterali (disgiunzione di letterali) \( (\{\neg A,B,C\},\{B,C,D\}) \)
 		\end{itemize}
 		
 		\begin{enumerate}
 			\setlength\itemsep{0.1mm}
 			\item Portare in forma normale congiuntiva es: \( (A\lor B \lor \neg C) \land (B \lor D \lor \neg A)\) (or tra lettere e and tra gruppi)
 			\item Convertire a letterali e clausole es: \( \{A,B,\neg C\},\{B,D,\neg A\}\) (ogni parentesi diventa una clausola con i propri letterali dentro)
 			\item L'obiettivo è raggiungere la clausola vuota, abbinado una clausola con un'altra ed eliminando IL letterale che in una è normale e nell'altra è negato
 		\end{enumerate}
 	 	
 	 	\section{Logica del primo ordine}
 	 	
 		\subsection{sintassi}
 		\begin{itemize}
 			\setlength\itemsep{0.1mm}
 			\item Lettere predicative: \(D(x,y) = 0/1\) falso o vero (es uguaglianza)
 			\item Lettere funzionali: \(P(x,y) = x\cdot y\) risultato della funzione (es moltiplicazione)
 			\item variabili/ costanti (es x,y/a,b)
 			\item connettivi soliti
 			\item quantificatori: \(\exists, \forall\)
 		\end{itemize}
 		\textbf{Per chiudere una formula del primo ordine si quantifica ogni variabile libera con il \(\forall\)}\\
 		\textbf{Forma normale prenessa}\\
 		Sposto tutti i quantificatori in testa (dopo aver chiuso la formula)\\\\
 		\textbf{Forma di skolem}\\
 		- la formula non deve più contenere \(\exists\)\\
 		- sostituisco le variabili precedute da \(\exists\) con tante lettere funzionali quanti \(\forall\) ci sono prima del \(\exists\) che devo togliere (le variabili che uso sono quelle dei \(\forall\) precedenti al \(\exists\) che ho tolto)\\
 		\subsection{equivalenze}
 		\begin{itemize}
 			\setlength\itemsep{0.1mm}
 			\item \(\neg \forall x A(x) = \exists x\neg A(x)\)
 			\item \(\neg \exists x A(x) = \forall x\neg A(x)\)
 			\item \(\forall A(x) \land B = \forall y (A(y) \land B(y))\) 
 			\item (vale anche per \(\exists\) e anche per \(\lor\)) (estraendo un quantificatore da \(\lor\) o \(\land\) non lo cambio) (si rinomina la variabile per sicurezza)
 			\item \(\forall x A(x) \implies B = \exists y (A(y) \implies B)\)
 			\item \(\forall x B \implies A(X) = \forall y (B \implies A(y))\) 
 			\item estraendo un quantificatore da un \(\implies\) si cambia se lo si estrae dal primo termine, non si cambia se lo si estrae dal secondo termine
 		\end{itemize}
 		\subsection{Forma a clausole}
 		\(\forall x_1,..., \forall x_n ((L_1\lor L_2 \lor L_3)\land (...) \land  ...)\)
 		\section{SPASS}
 		\subsection{struttura di un programma spass}
 		list\_of\_symbols.\\
 		\tab functions[(n\_funz,arità),...,(cost,0)].\\
 		\tab predicates[(n\_predicato,arità),...].\\
 		end\_of\_list.\\\\
 		list\_of\_formulae(axioms).\\
 		\tab formula(...).
 		end\_of\_list.\\\\
 		list\_of\_formulae(conjectures).\\
 		\tab congettura\_da\_verificare(...).
 		end\_of\_list.
 		\subsection{sintassi}
 		\begin{itemize}
 			\setlength\itemsep{0.1mm}
 			\item \(\land\) = and(), \(\lor\) = or(), \(\neg\) = not()
 			\item \(\implies\) = implies(), \(\iff\) = equiv()
 			\item \(\forall\) = forall([x],...)
 			\item \(\exists\) = exists([x],...)
 		\end{itemize}
 		\textbf{spass lavora solo su formule chiuse}\\\\
 		\textbf{funzioni:}
 		ad esempio moltiplicazione (le costanti sono funzioni di arità 0)\\
 		\textbf{predicati:}
 		ad esempio uccide, è presente, è incantato, commercia\\
		\end{multicols*}
	
	
		\begin{multicols*}{3}
			
			
		\section{Algebra}
		\subsection{Strutture algebriche}
		\begin{itemize}
			\item Le strutture algebriche sono una coppia (A,\(\Omega\)) Dove \(\Omega={\omega_1,...,\omega_K}\) è un insieme di operazioni interne all'insieme A
		\end{itemize}
		\textbf{tipi di strutture algebriche}
		\begin{itemize}
			\item \textbf{semigruppo} (A,\(\cdot\)) Dove \(\cdot\) è un'operazione binaria che soddisfa la proprietà associativa (se l'operazione è commutativa il semigruppo si dice semigruppo commutativo)
			\item \textbf{Monoide} (M,*,e) Dove (M,*) è un semigruppo e e\(\in\)M è un elemento neutro (unico) all'operazione *
			\item \textbf{Gruppo} (G,*,e,\(^{-1}\)) Dove (G,*,e) è un monoide ed esiste l'inverso \(\forall g\in G \)  \(\exists  h\in G \) tale che g*h=h*g=e (h è l'inverso destro e sinistro di g)
			\item \textbf{Anello} (A,+,\(\cdot\)) Dove (A,+) è un gruppo commutativo con elementro neutro 0, e (A,\(\cdot\)) è un semigruppo
			\item \textbf{Corpo e campo} Un corpo è un anello (A,+,\(\cdot\),1) con identità tale che (A\textbackslash \{0\},\(\cdot\)) è un gruppo, se questo gruppo è commutativo si parla di campo
		\end{itemize}
		\textbf{Zero di un sermigruppo (S,\(\cdot\)) (elemento assorbente)}\\
		è un elemento \(z\in S \) tale che \(\forall s \in S\)\\
		\(s\cdot z = z \cdot s = z\)\\
		\textbf{Divisori dello zero}\\
		In un anello (A,+,\(\cdot\)) due elementi a,b a\(\neq\)0,b\(\neq\)0 si dicono divisori dello zero se a\(\cdot\)b=0\\
		In un anello privo di divisori dello zero valgono le leggi di cancellazione a sinistra e destra\\
		\textbf{Osservazione}\\
		Se il semigruppo moltiplicativo (A\textbackslash \{0\},\(\cdot\)) è un gruppo \(\implies\) l'anello non ha divisori dello zero\\
		
		\textbf{Quaternioni}:corpo che non è un campo\\
		Definiti da: \(H=\{a\cdot i+b\cdot j+c\cdot k+d \:\:\:\: a,b,c,d\in \Re \}\)
		
		\subsection{Sottostrutture}
		Data (A,\(\Omega\)) struttura algebrica e H\(\subseteq\)A, (H,\(\Omega\)) è una sottostruttura algebrica se tutte le operazioni di omega "si restringono" ad H:\\
		\(*\in \Omega \:\:\: \forall h_1,h_2\in H \:\:\: h_1*h_2\in H\)\\
		Quindi tutte le operazioni \(\Omega\) sono chiuse in H\\
		\begin{enumerate}
			\item (H,\(\cdot\)) è sottosemigruppo di un semigruppo (S,\(\cdot\)) (H\(\subseteq\) S) \(\iff\) \(\forall a,b \in H \:\:\: a\cdot b\in H \\ \cdot : H\times H \rightarrow H\)
			\item (H,\(\cdot\),e) è sottomonoide del monoide (M,\(\cdot\),e) \(\iff\) è un sottosemigruppo \(\land \:\: e \in H\)
			\item (H,\(\cdot\),e, \(^{-1}\)) è un sottogruppo del gruppo (G,\(\cdot\),e, \(^{-1}\)) \(\iff\)\\ \(\forall a,b \in H \:\:\: a\cdot b \in H\) \\
			\(\forall a \in H \:\:\: a^{-1} \in H\)\\\\
			\textbf{Criterio per gruppi:}\\ (H,\(\cdot\)) è sottogruppo \(\iff\) \\
			\(\forall a,b \in H \:\:\: a\cdot b^{-1} \in H\)
			\item (H,+,\(\cdot\)) è un sottoanello di (A,+,\(\cdot\)) \(\iff\):\\
			(H,+) è un sottogruppo di (A,+)\\
			(H,\(\cdot\)) è un sottosemigruppo di  (A,\(\cdot\))
			\item (H,+,\(\cdot\)) sottocampo/sottocorpo di (A,+,\(\cdot\)) se è un sottoanello e (H \textbackslash \{0\},\(\cdot\)) è un sottogruppo di (A \textbackslash \{0\},\(\cdot\))
		\end{enumerate} 
		
		\textbf{Strategia: uso i criteri delle sottostrutture tramite strutture note}\\\\
		Strutture note:
		\begin{itemize}
			\item Campi:\( (\mathbb{Z}/5,+,\cdot),(\mathbb{Z},+,\cdot),(\mathbb{Q},+,\cdot),(\mathbb{R},+,\cdot),(\mathbb{C},+,\cdot)\)
			\item Anelli: \((\mathbb{R}[x],+,\cdot)\) (polinomi in x), \((\mathbb{M}_{nn}(\mathbb{R}),+,\cdot)\) (matrici quadrate), \((\mathbb{Z}/5,+,\cdot)\) (classi di equivalenza per numeri non primi (anelli con divisori deello zero))
			\item Gruppi: \((GL_n(\mathbb{R}),+)\) (matrici con determinante \(\neq\) 0)
			\item Monoidi: \((\mathbb{N},+,0)\) 
		\end{itemize} 
		\subsection{Congruenza/strutture quoziente/omomorfismi}
		Data una struttura algebrica (A,\(\Omega\)) una relazione \(\rho \subseteq A\times A\) di equivalenza si dice \textbf{compatibile} per * \(\in \Omega\) se: \\
		\(\forall a_1,a_2,b_1,b_2 \:\:\:\:\: a_1 \rho b_1 = a_2\rho b_2 \implies (a_1*a_2)\rho (b_1*b_2) \)\\
		Se \(\rho\) è compatibile con tutte le operazioni di \(\Omega\) si chiama \textbf{congruenza}\\
		
		
		Data (A,\(\Omega\)) struttura e \(\rho \subseteq A\times A\) congruenza allora per ogni operazione *\(\in \Omega\) possiamo definire una nuova operazione interna A\textbackslash\(\rho\)\\
		\(*_{\rho} : A \setminus \rho \times A \setminus \rho \rightarrow A \setminus \rho\)\\
		Definita da \([a]_\rho *_\rho [b]_\rho := [a*b]_\rho\)\\
		Nuova struttura algebrica: \((A\setminus \rho ,\Omega_\rho)\) dove \(\Omega_\rho = \{*_\rho \:\:\: *\in \Omega\}\)\\
		
		\textbf{Omomorfismo}\\
		Un omomorfismo è una funzione \(f\) che preserva tutte le operazioni \(\Omega_1 \) e \(\Omega_2 \) tra le strutture \((A_1, \Omega_1)\) e \((A_2, \Omega_2)\)\\
		Tipi di omomorfismo in base a \(f\):
		\begin{itemize}
			\item \(f\) iniettiva \(\rightarrow\) monomorfismo
			\item \(f\) suriettiva \(\rightarrow\) epimorfismo
			\item \(f\) biunivoca \(\rightarrow\) isomorfismo			
		\end{itemize}
		\textbf{Criterio per gruppi}\\
		Dati (G,*) e (H,\(\cdot\)) gruppi \(f:G\rightarrow H\) è un omomorfismo \(\iff\) \(\forall g_1,g_2 \:\:\:\: f(g_1,g_2) = f(g_1) \cdot f(g_2)\)
		\textbf{Criterio per anelli}\\
		Dati (A,+,\(\cdot\)) e (B,\(\oplus\),\(\odot\)) anelli \(\phi : A\rightarrow B\) è un omomorfismo se:\\
		\(\forall a,b \in A \:\:\:\: \varphi(a+b) = \varphi(a) \oplus \varphi (b) \)\\
		\(\forall a,b \in A \:\:\:\: \varphi(a\cdot b) = \varphi(a) \odot \varphi (b) \)\\
		\subsection{Sottogruppi normali(gruppi)/ ideali(anelli)}
		Un sottogruppo H di un gruppo (G,*) si dice normale se:\\
		\(\forall g \in G, \forall h \in H \:\:\:\:\: g^{-1} *h*g \in H \:\:\: (\iff \forall g\in G \:\:\: g^{-1}*h*g \subseteq H)\)\\
		Osservazione: se G è commutativo \(\implies\) tutti i sottogruppi sono normali
		\(g^{-1}*h*g=g^{-1}*g*h=h*e_G=h\in H \:\:\: \forall h \in h \:\: \forall g \in G \)\\
		Proposizione: se \(\rho\) è una congruenza del gruppo (G,*) allora \([e_G]_\rho\) è un sottogruppo normale\\\\
		Un \textbf{ideale} I di un anello (A,+,\(\cdot\)) è un sottoanello di A che soddisfa l'assorbimento \(\forall a \in A\):\\
		Destra: I\(\cdot\)A = \(\{x\cdot a: x\in I\}\subseteq\) I\\
		Sinistra: I\(\cdot\)A = \(\{a \cdot x: x \in  I\} \subseteq\) I\\
		
		
		
		
		
		
		\end{multicols*}
\end{document}