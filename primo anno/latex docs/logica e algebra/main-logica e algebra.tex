\documentclass{article}
\usepackage[utf8]{inputenc}
\usepackage{amsmath,amssymb}
\usepackage{graphicx}
\usepackage[a4paper,margin=1.2cm]{geometry}
\usepackage{multicol}
\graphicspath{ {./img/} }

\DeclareMathOperator{\ima}{Im}


\begin{document}
	
		
		\section{Proprietà relazioni}
		{\Large\[\text{Proprietà di una relazione binaria  } R\subseteq A \times A\]}
		\\\\
		{\large Seriale}\\\\
		\(\forall a \in A \  \exists b \in A (a,b) \in R\)\\\\
		Grafo: ogni vetice ha una freccia uscente\\
		Matrice: ogni riga ha almeno un "1"\\\\
		{\large Riflessiva}\\\\
		\(\forall a \in A \  (a,a) \in R\)\\\\
		Grafo: ogni vetice ha un cappio\\
		Matrice: sulla diagonale ho tutti "1"\\\\
		{\large Simmetrica}\\\\
		\(\forall a,b \in A \  (a,b) \in R \Rightarrow (b,a)\in R\)\\\\
		Grafo: Ogni freccia in una direzione ne ha una della direzione opposta\\
		Matrice: \(Mr=Mr^T\)\\\\
		{\large Antisimmetrica}\\\\
		\(\forall a,b \in A \  se\ (a,b) \in R \ e\  (b,a)\in R \Rightarrow a=b\)\\\\
		Grafo: Eccetto i cappi, per ogni freccia non ce ne deve essere una uguale ma nella direzione opposta\\
		Matrice: eccetto la diagonale, se in pos (i,j) c'è un 1, allora in posizione (j,i) ci deve essere 0\\\\
		{\large Transitiva}\\\\
		\(\forall a,b,c \in A \  (a,b) \in R e\ (b,c)\in R \Rightarrow (a,c)\in R\)\\\\
		Grafo: se a è collegato a b e b è collegato a c anche a deve essere collegato a c \\\\\\\\
		
		{\large Osservazioni}
		\begin{itemize}
			\item seriale \(\nRightarrow\) riflessiva
			\item antisimmetrica \(\nRightarrow\) non simmetrica
			\item transitiva + simmetrica \(\nRightarrow\) riflessiva
			\item riflessiva \(\Rightarrow\) seriale
			\item transitiva + simmetrica + seriale \(\Rightarrow\) riflessiva
		\end{itemize}
		\ \\\\
		
		{\Large Relazioni di equivalenza}\\\\
		{\large Una relazione si dice di equivalenza se è riflessiva, transitiva, simmetrica}\\\\\\
	
		{\Large Relazioni d'ordine}\\\\
		{\large Una relazione si dice d'ordine se è riflessiva, transitiva, antisimmetrica}\\\\\\\\
		
		{\Large Elementi estremali}\\\\
		\begin{itemize}
			\item Massimo: \(se \ \forall \ x \in A \ a\leq x\)
			\item Minimo: \(se \ \forall \ x \in A \ a\geq x\)
			\item Minimale: \(\forall \ x \in A \ se\ x \leq a \Rightarrow x=a\)
			\item Massimale: \(\forall \ x \in A \ se\ x \geq a \Rightarrow x=a\)
		\end{itemize}
	 	\ \\
	 	{\large Oss: Un minimo è minimale, un massimo è massimale}\\\\
	 	
	 	Fare da maggiorante in poi
		
		
		
		
		
		
\end{document}