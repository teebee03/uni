\documentclass{article}
\usepackage[utf8]{inputenc}
\usepackage{amsmath,amssymb}
\usepackage{graphicx}
\usepackage[a4paper,margin=1.2cm]{geometry}
\usepackage{multicol}
\graphicspath{ {./img/} }

\DeclareMathOperator{\ima}{Im}


\begin{document}
	
		
	 	\section{Doppia rampa}
	 	\(V_x \cdot T_{up} = |Vr| \cdot T_{down}\)
	 	\begin{itemize}
	 		\item Tup e Tdown = tempo di salita e discesa rampa
	 		\item Vr = tensione di riferimento interna
	 		\item Vx = tensione da misurare
	 	\end{itemize}
 		\
 		\(Numero cicli = \frac{T_{down}}{T_{clock}}\)\\\\
 		\(T_{up} = T_{clock} \cdot N_{up} \)\\\\
 		\(T_{down} = T_{clock} \cdot N_{down} \)\\\\
 		\(\Delta V = \frac{|Vr|}{N_{up}}\)\\\\\\
 		Per eliminare i disturbi il tempo di integrazione (Tup) deve essere un multiplo intero dei periodi dei disturbi\\\\
 		
 		l'incertezza dello strumento è legato alla capacità di conteggio (Nup e Ndown) e a Vr\\\\
 		\(u(Vx) = \sqrt{u_r^2(N)+u_r^2(Vr)}\)\\\\
 		\(u_r(N) = \frac{u(N)}{N}\)\\\\
 		\(u(N) = \frac{1}{\sqrt{12}}\)\\\\\\
 		
 		{\large Progettazione doppia rampa}
 		
 		\begin{itemize}
 			\item trovare Tup (in base ai disturbi da sopprimere)
 			\item trovare Nlivelli (tramite Tup) (Ndown sarà calcolato con Ndownmax usando Vx = portata)
 			\item trovare frequanza di clock usando la risoluzione
 		\end{itemize}
 		\(\Delta V \cdot T_{up} = |Vr| \cdot T_{clock}\)\\\\
 		Per trovare la costante di tempo dell'integratore (RC) (sapendo Vo = minimo valore all'uscita dall'integratore)\\\\
 		\(Vo = -Vx \cdot \frac{T_{up}}{RC}\)
 		Tempo di conversione = Tup + Tdown dove Tup costante e Tdown variabile\\\\\\
 		
 		\section{Voltmetro SAR (approssimazioni successive)}
 		
 		\(\Delta V = \frac{D_{adc}}{2^N}\)\\\\
 		Con Dadc = dinamica, N = numero di bit, e deltaV = risoluzione dimensionale (risoluzione adimensionale = nbit oppure \(1/2^n\))\\\\\\
 		
 		incertezza di quantizzazione = \(\frac{\Delta V}{\sqrt{12}}\)\\\\\\\\\\\\
 		 		
 		Numero di Bit equivalenti\\\\
 		
 		\(n_e = n-\frac{1}{2}log_2(1+\frac{\sigma^2_c}{\sigma^2_q})\)\\\\
 		(sigma n = varianza dei rumori = varianza dei rumori esterni + varianza dei rumori interni)\\\\
 		(sigma q = varianza di quantizzazione)\\\\
 		\(\sigma_{N,int}^2 = V_{N,eff}^2\)\\
 		\(\sigma_{q}^2 = \frac{\Delta V^2}{12}\)\\
 		
 		
 		
 		
 		\section{Oscilloscopi}
 		
 		\begin{itemize}
 			\item Fino a 1Khz si usa chopped oltre alternated (visualizzazione multitraccia)
 			\item ugabuga
 		\end{itemize}
 		\
 	
 	 	\(t_{salita} MISURATO = \sqrt{t^2_{updevice}+t^2_{uposcilloscopio}}\)\\\\
 		\(t_{uposcilloscopio} = \frac{0,35}{bandaosc}\)\\\\
 		
		
\end{document}